\documentclass{article}
\usepackage[margin=0.5in]{geometry}
\usepackage[utf8]{inputenc}
\usepackage{textcomp}
\usepackage{polski}

\usepackage{pgfplots}
\pgfplotsset{width=18cm,compat=1.9}
%\usepgfplotslibrary{external}
%\tikzexternalize

\begin{document}

\section*{Porównanie wydajności drzew van Emde Boasa, x-fast oraz y-fast}
Mateusz Lewko \\

\textbf{Wykonane prace} Zaimplementowałem trzy struktury danych:
\begin{itemize}
  \item drzewo van Emde Boasa (\texttt{v-eb.h}),
  \item x-fast trie (\texttt{x-fast.h}),
  \item y-fast trie wykorzystujące x-fast oraz drzewo bst (\texttt{y-fast.h}),
  \item oraz wrapper na std::set (dla porównania, jest w pliku (\texttt{test.h})).
\end{itemize}
Każda z nich implementuje taki sam interfejs (lookup, successor,
insert --- plik (\texttt{interface.h})).
Konstruktor każdej struktury przyjmuje wykładnik $k$ rozmiaru uniwersum $U = 2^k$.
Jeśli w strukturze nie ma następnika dla danego zapytania, to zwracane jest $U$.
Zarządzanie pamięcią jest zapewnione przez użycie shared pointerów. Skutkiem ubocznym
użycia shared pointerów jest duża stała wykonywanych operacji.

Poprawność zaimplementowanych metod jest sprawdzana poprzez wypisanie xora
wyników operacji \textit{successor} dla każdej struktury.

\textbf{Testy wydajności}
Zaimplementowałem dwa typy testów. W pierwszym, dla danego $N$ losowane jest
$N$ liczb jako wejście do operacji insert, a następnie $N$ liczb na 
których kolejno wykonywane są zapytania o następnika.

\textbf{Kompilacja} Wszystkie testy można skompilować jednym poleceniem w
systemie linux: \texttt{g++ test.cpp -o test -O3 -Wall -Wextra -std=c++17},
a następnie uruchomić: \texttt{./test}.


\begin{figure}
  \centering
  \begin{tikzpicture}
    \begin{semilogxaxis}[
        title={Test 1. N insertów, a następnie N zapytań o następnika},
        xlabel={Wartość N},
        ylabel={Czas działania w sekundach},
        xmin=100000, xmax=12000000,
        ymin=0, ymax=60,
        xtick={100000, 1000000, 2000000, 4000000, 8000000, 12000000},
        ytick={1, 5, 10, 15, 30, 45, 60 },
        legend pos=north west,
        ymajorgrids=true,
        grid style=dashed,
      ]


      \addplot[
        color=blue,
        mark=square,
      ]
      coordinates {
          (100000, 0.04)(1000000, 0.62)(2000000, 1.4)(4000000, 2.66)(8000000, 4.89)
        };

      \addplot[
        color=red,
        mark=square,
      ]
      coordinates {
          (100000, 0.37)(1000000, 3.81)(2000000, 6.69)(4000000, 11.18)(8000000, 17.74)
        };

      \addplot[
        color=green,
        mark=square,
      ]
      coordinates {
          (100000, 0.24)(1000000, 4.4)(2000000, 9.04)(4000000, 17.92)(8000000, 33.27)
        };

      \addplot[
        color=brown,
        mark=square,
      ]
      coordinates {
          (100000, 0.03)(1000000, 0.88)(2000000, 2.79)(4000000, 6.73)(8000000, 15.39)
        };

      \legend{vEB, x-fast, y-fast, std::set}

    \end{semilogxaxis}
  \end{tikzpicture}
\end{figure}

\end{document}
